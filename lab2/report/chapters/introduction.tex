% !TeX root = ../main.tex
\chapter{Einführung}

In diesem Laborversuch soll die Anwendung der klassischen Bildverarbeitung anhand eines praktischen Anwendungsbeispiels erprobt werden. Zu implementieren ist ein Inspektionssystem zur Überprüfung von Solarzellen. Diese werden zunächst unter Ausnutzung des Elektroluminiszenzeffekts zum Leuchten gebracht und mittels Infrarotkamera abfotografiert. Aus einem so erhaltenen manuell gelabelten Teildatensatz wird eine klassische Bildverarbeitungskette inklusive Klassifikator implementiert. Abschließend wird diese zur Verifikation auf die verbleibenden Bilder des Datensatzes angewandt.

Im ersten Laborteil wird zunächst die Aufnahme der Elektroluminiszenzbilder durchgeführt. Der zweite Teil beinhaltet die Ausarbeitung der Bildverarbeitungskette in verschiedenen Variationen. Als Vorbereitung gilt es die Bildverarbeitungskette in einem Ablaufdiagramm konzeptionell durchzuplanen und sich mit den Bausteinen der klassischen Bildverarbeitung enger vertraut zu machen. Dies ist kein Bestandteil dieses Berichts und wird an dieser Stelle nicht weiter ausgeführt.

Alle in diesem Laborversuch entstandenen Quelltexte werden vollständig und kommentiert im Anhang wiedergegeben.
