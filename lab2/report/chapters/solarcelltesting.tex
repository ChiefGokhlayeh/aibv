\chapter{Solarzellenprüfung}

\section{Elektrolumineszenzprüfung von Solarzellen}

Im ersten Aufgabenteil des Labors sind Elektrolumineszenzbilder mit dem Solarzellentester aufzunehmen. Das Gerät arbeitet dabei weitgehend automatisch. Die Zellen werden vorsichtig von Staub bereinigt und in die ausziehbare Halterung eingelegt. Es ist darauf zu achten, dass die Solarzelle mit der am oberen Ende der Halterung befindlichen Leiste kontaktiert, worüber ein Strom in die Zellen gespeist wird. Nach Einschieben der Halterung werden die Bilder automatisch aufgenommen und in der angeschlossenen Bediensoftware angezeigt. Von dort können die Bilder per Knopfdruck mit aktueller Uhrzeit und Datum versehen abgespeichert werden.

\subsection{Fehlerklassen und Ursachen}

Einige Fehlerklassen, die auch Teil des zweiten Laborteils sind, wurden bereits in \cref{fig:example_defects} exemplarisch aufgezeigt. An dieser Stelle soll daher der Fokus auf mögliche Ursachen dieser Defekte gelegt werden.

In~\cite{Köntges2014,Deitsch2018} werden einige typische Fehlerklassen beschrieben. Demnach sind die als \defectdarkarea{} bezeichneten Dunkelstellen auf elektrisch degradierte oder komplett isolierte Zellen zurückzuführen. Defekte wie \defectfinger{} werden unter anderem durch Unregelmäßigkeiten im Wafer-Substrat verursacht, können aber auch durch Brüche an den Busverbindungen entstehen. Brüche bzw.\@ \defectcrack{}(s) entstehen vorwiegend während der mechanischen Verarbeitung der Wafer. Darunter fällt bspw.\@ das Sägen der Wafer oder das \emph{\foreignlanguage{english}{Stringing}}, also das Zusammenlegen, Verlöten und Einbetten mehrerer Zellen in ein Glas-Panel. % chktex 36

\subsection{Vermutete Auswirkungen}

Es liegt nahe, dass Dunkelstellen (\defectdarkarea{}) die größten Auswirkungen auf die Leistungsfähigkeit der Zelle haben. Bereich die während der Elektrolumineszenzprüfung nicht oder nur schwach aufleuchten, werden auch im späteren Betrieb nicht die gewünschte Leistung erbringen. Bruchstellen (\defectcrack{}), abgesehen von solchen die eine Busverbindungen trennen, haben zunächst keine nennenswerte Auswirkung auf die Leistungsfähigkeit. Sie können sich jedoch durch zyklische Temperaturwechsel während Nutzung der Zelle vergrößern und zu größeren Ausfällen führen~\cite{Deitsch2018}. Ausfälle von Teilen einzelner Finger (\defectfinger{}) haben laut~\cite{Köntges2014} keine größeren Auswirkungen auf Leistungsfähigkeit und Langlebigkeit, und werden als weniger problematisch angesehen.

Ein hier nicht näher betrachteter Defekt steht im Zusammenhang mit Inhomogenitäten beim Brennen der Zelle. Diese führen zu weichgezeichneten Dunkelstellen ohne harte Kanten im mittleren Bereich der Zelle. Außerdem kann Stressbildung beim Löten zum Ausfall eines kompletten Fingers zwischen zwei Busverbindungen führen. Diese und weitere Defektklassen sind in~\cite{Köntges2014} näher beschrieben.
