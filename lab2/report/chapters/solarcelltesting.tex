\chapter{Solarzellenprüfung}

\section{Elektrolumineszenzprüfung von Solarzellen}

Im ersten Aufgabenteil des Labors sind Elektrolumineszenzbilder mit dem Solarzellentester aufzunehmen. Das Gerät arbeitet dabei weitgehend automatisch. Die Zellen werden vorsichtig von Staub bereinigt und in die ausziehbare Halterung eingelegt. Es ist darauf zu achten, dass die Solarzelle mit der am oberen Ende der Halterung befindlichen Leiste kontaktiert, worüber ein Strom in die Zellen gespeist wird. Nach Einschieben der Halterung werden die Bilder automatisch aufgenommen und in der angeschlossenen Bediensoftware angezeigt. Von dort können die Bilder per Knopfdruck mit aktueller Uhrzeit und Datum versehen abgespeichert werden.

\subsection{Fehlerklassen und Ursachen}

Einige Fehlerklassen, die auch Teil des zweiten Laborteils sind, wurden bereits in \cref{fig:example_defects} exemplarisch aufgezeigt. An dieser Stelle soll daher der Fokus auf mögliche Ursachen dieser Defekte gelegt werden.

In~\cite{Köntges2014,Deitsch2018} werden einige typische Fehlerklassen beschrieben. Demnach sind die als \defectdarkarea{} bezeichneten Dunkelstellen auf elektrisch degradierte oder komplett isolierte Zellen zurückzuführen. Defekte wie \defectfinger{} werden unter anderem durch Unregelmäßigkeiten im Wafer-Substrat verursacht, können aber auch durch Brüche an den Busverbindungen entstehen. Brüche bzw.\@ \defectcrack{}(s) entstehen vorwiegend während der mechanischen Verarbeitung der Wafer. Darunter fällt bspw.\@ das Sägen der Wafer oder das \emph{\foreignlanguage{english}{Stringing}}, also das Zusammenlegen, Verlöten und Einbetten mehrerer Zellen in ein Glas-Panel. % chktex 36

\subsection{Vermutete Auswirkungen}

Es liegt nahe, dass Dunkelstellen (\defectdarkarea{}) die größten Auswirkungen auf die Leistungsfähigkeit der Zelle haben. Bereich die während der Elektrolumineszenzprüfung nicht oder nur schwach aufleuchten, werden auch im späteren Betrieb nicht die gewünschte Leistung erbringen. Bruchstellen (\defectcrack{}), abgesehen von solchen die eine Busverbindungen trennen, haben zunächst keine nennenswerte Auswirkung auf die Leistungsfähigkeit. Sie können sich jedoch durch zyklische Temperaturwechsel während Nutzung der Zelle vergrößern und zu größeren Ausfällen führen~\cite{Deitsch2018}. Ausfälle von Teilen einzelner Finger (\defectfinger{}) haben laut~\cite{Köntges2014} keine größeren Auswirkungen auf Leistungsfähigkeit und Langlebigkeit, und werden als weniger problematisch angesehen.

Ein hier nicht näher betrachteter Defekt steht im Zusammenhang mit Inhomogenitäten beim Brennen der Zelle. Diese führen zu weichgezeichneten Dunkelstellen ohne harte Kanten im mittleren Bereich der Zelle. Außerdem kann Stressbildung beim Löten zum Ausfall eines kompletten Fingers zwischen zwei Busverbindungen führen. Diese und weitere Defektklassen sind in~\cite{Köntges2014} näher beschrieben.

\section{Automatisierte Optische Inspektion}

Zum Zwecke der automatischen Inspektion der mittels Elektrolumineszenzprüfung gewonnenen Bilder bedarf es einer Fehlersegmentierung und anschließenden Klassifizierung in die genannten Defektklassen. Jedem Bild wird dabei nur eine einzelne Klasse zugeordnet. Eine Arbitrierung zwischen gleichzeitig auftretenden Defekten findet über das flächenmäßig größte erkannte Segment statt.

Der grundlegende Aufbau der Bildverarbeitungskette ist in \cref{fig:overall_flow_chart} gezeigt. Da einige der Prozessschritte bereits vorbereitet waren, liegt der Fokus dieser Ausarbeitung auf den Blöcken \textbf{Segmentierung}, \textbf{Merkmalsextraktion}, \textbf{Training} und \textbf{Klassifikation}. Im Zuge des Labors wurden jeweils zwei Varianten für Segmentierung und Merkmalsextraktion implementiert. Auf diese wird in den folgenden Abschnitten näher eingegangen. Die Klassifikation findet mittels \foreignlanguage{english}{\(k\)-nearest neighbor} Verfahren statt. In dessen Training wird zu den ausgewählten Merkmalsvektoren jeweils die gelabelte \foreignlanguage{english}{\emph{Ground Truth}} vorgemerkt. Zusätzlich ist der Merkmalsvektor \(\boldsymbol{v}\) zu normieren. Hierbei wurde eine Normierung zum komponentenweisen Mittelwert \(\bar{\boldsymbol{v}}\) und zur komponentenweisen Standardabweichung \(\boldsymbol{\sigma}_{\boldsymbol{v}}\) nach der Formel,
%
\begin{equation*}
    \boldsymbol{v}_{\mathrm{norm}} = {\left( \boldsymbol{v} - \bar{\boldsymbol{v}} \right)} \oslash \boldsymbol{\sigma}_{\boldsymbol{v}} {,}
\end{equation*}
%
gewählt, mit dem Operator \({\left \{\cdot \right \}} \oslash {\left \{\cdot \right \}}\) definiert als Hadamard oder elementweise Division zweier gleich dimensionierter Vektoren. Im Klassifikationsschritt mit Validierungsdaten wird dann lediglich die gleiche Segmentierung, Merkmalsextraktion und Normierung mittels \(\bar{\boldsymbol{v}}\) und \(\boldsymbol{\sigma}_{\boldsymbol{v}}\) durchgeführt. Für jeden Vektor \(\boldsymbol{v}_{\mathrm{val}}\) aus dem Validierungsdatensatz wird die Euklidische Distanz zur Menge aller Merkmalsvektoren \({\left \{ \boldsymbol{v}_{\mathrm{train} } \right \}}\) aus dem Trainingsschritt berechnet. Um eine Vorhersage zu treffen werden die \(k\) kleinsten Distanzen ausgewählt und die Klassenzugehörigkeit der entsprechenden Trainingsvektoren bestimmt. Ein Mehrheitsentscheid der \(k\) entnommenen Klassen dient der finalen Entscheidungsfindung. Bei Parität wird die Klassenzugehörigkeit des in der Trainingsreihenfolge ersteren Merkmalsvektors angenommen. Dabei handelt es sich nicht um eine aktiv getroffene Entscheidung, sondern um einen Nebeneffekt der verwendeten Python-Bibliothek. Bei der Hyperparametrierung ist darauf zu achten, dass \(k\) ungerade ist. In allen ausprobierten Varianten wurde festgestellt, dass die höchste \foreignlanguage{english}{Accuracy} mit \(k = 1\), sprich einem einfachen \foreignlanguage{english}{nearest neighbor} Algorithmus, erzielt werden konnte. Auf verschiedene Varianten der Segmentierung und Merkmalsextraktion wird nun näher eingegangen.

\begin{figure}[ht]
    \centering
    \begin{tikzpicture}[
        start/.style={
                font=\small,
                draw,
                very thick,
                shape=ellipse,
                text depth=0.35ex,
                inner sep=0.25cm,
                align=center,
                on chain,
            },
        end/.style={
                start,
                join,
            },
        process/.style={
                draw,
                very thick,
                rounded corners,
                text depth=0.35ex,
                inner sep=0.25cm,
                align=center,
                on chain,
                join,
            },
        provided/.style={
                font=\small,
            },
        every join/.style={very thick, -{Stealth[]}},
        ]
        \begin{scope}[start chain=1 going below, node distance=5mm]
            \node [start] {Start};
            \node [process, provided] {Traningsbilder einlesen};
            \node [process] {\textbf{Segmentierung}\\\small BLOBs der Defekte extrahieren};
            \node [process] {\textbf{Merkmalsextraktion}\\\small Merkmale aus Segmenten extrahieren};
            \node [process, provided] {Darstellung des Merkmalraums};
            \node [process] {\textbf{Training}\\\small Erstellen des k-nearest Neighbor Raums};
            \node [process, provided] {Validierungsbilder einlesen und segmentieren};
            \node [process] {\textbf{Klassifikation}\\\small Bilder klassifizieren über \(k\)-nearest Neighbor};
            \node [process, provided] {Bestimmung der Accuracy und Confusion Matrix};
            \node [end] {Ende};
        \end{scope}
    \end{tikzpicture}
    \caption{Grundlegender Aufbau der Bildverarbeitungskette.\label{fig:overall_flow_chart}}
\end{figure}

\subsection{Variante 1}

\Cref{fig:variant_1} zeigt den schematischen Ablauf von Variante 1. Zunächst wird eine Überbelichtung des Grauwertbildes vorgenommen. Dabei dient das \(P_{70\%}\) Perzentil als neuer Maximalwert. Wie bei einer Überbelichtung üblich werden alle Werte über \(P_{70\%}\) bei \num{255} abgeschnitten. Als Nächstes findet eine Belichtungskorrektur statt. Diese basiert auf Beobachtungen aus Stichproben, bei denen die obere rechte Ecke stets etwas dunkler als die anderen schien. Dies stellte sich in der Laborvorbereitung als limitierender Faktor bei globaler Schwellwertbildung heraus. Als Korrektur wird eine gaußförmige Helligkeitsmatrix auf die obere Rechte Ecke addiert. Das so normierte Grauwertbild dient nun als Eingang für vier nachgelagerte Prozessschritte. Zum einen wird eine adaptive Schwellwertbildung durchgeführt, um lokale Dunkelstellen (Defekte, Schmutz und konstruktionsbedingte Strukturen) zu extrahieren. Letztere müssen noch vom so teil-segmentierten Binärbild abgezogen werden. Um Makrostrukturen wie Busleisten zu eliminieren wird eine Maske generiert. Dazu wird das normierte Grauwertbild zunächst mittels gaußscher Weichzeichnung tiefpassgefiltert. Anschließend wird eine \foreignlanguage{english}{Canny Edge-Detection} auf das Bild angewandt, mit der die Kanten an den Busleisten extrahiert werden. Durch Korrelation mit den \foreignlanguage{english}{Edge-Finder} Kernels
%
\begin{align*}
    \boldsymbol{S}_{\mathrm{edge},\mathrm{vert}}^{\phantom{\intercal}} & = \left[\phantom{
            \begin{matrix}
                0      \\
                \vdots \\
                0      \\
            \end{matrix}}
        \right.\hspace{-0.5em}
        \underbrace{\begin{matrix}
                0 & \hdots & 0 \\
                  & \vdots &   \\
                0 & \hdots & 0 \\
            \end{matrix}}_{N \times 15}
        \quad\underbrace{\begin{matrix}
                1 & \hdots & 1 \\
                  & \vdots &   \\
                1 & \hdots & 1 \\
            \end{matrix}}_{N \times 15}
        \quad\underbrace{\begin{matrix}
                0 & \hdots & 0 \\
                  & \vdots &   \\
                0 & \hdots & 0 \\
            \end{matrix}}_{N \times 15}
        \hspace{-0.5em}
        \left.\phantom{
            \begin{matrix}
                0      \\
                \vdots \\
                0      \\
            \end{matrix}
    }\right] \quad                                                     & N \times 45 {,}\                                                 & \text{und}        \\
    \boldsymbol{S}_{\mathrm{edge},\mathrm{horz}}^{\phantom{\intercal}} & = \boldsymbol{S}_{\mathrm{edge},\mathrm{vert}}^{\intercal} \quad & 45 \times N {,} &
\end{align*}
%
wobei \(N\) der Höhe bzw.\@ Breite des quadratischen Bildes entspricht, werden die Pixelkoordinaten vom linken, rechten, oberen und unteren Rand sowie den drei Versorgungsleisten extrahiert. Um diese wird nun die Bildmaske gelegt. Dies ermöglicht relativ robustes maskieren der Busleisten. Da die adaptive Schwellwertbildung nur eine lokale Grauwertverteilung berücksichtigt, gelangt auch Schmutz und Feinstrukturen in die Teilsegmentierung. Außerdem scheitert der lokale Ansatz an größeren Dunkelstellen (\defectdarkarea{}), was dazu führt, dass diese aus der Segmentierung fallen. Um dies zu verhindern wird eine globale Schwellwertbildung mit verhältnismäßig niedrigem Schwellwert angewandt. Die so erkannten \texttt{\foreignlanguage{english}{OFF}} Pixel werden dem Teilsegmentierung hinzugefügt. Gleiches wird mit einem verhältnismäßig hohen Schwellwert für \texttt{\foreignlanguage{english}{ON}} Pixel vorgenommen um Schmutzpartikel und andere Störungen von der Teilsegmentierung zu entfernen. Nach Kombination der Bildmasken mit dem teil-segmentierten Bild wird final eine Erodierung vertikaler Kanten durchgeführt. Dazu wird ein der Kernel \(\boldsymbol{S}_{\mathrm{erode},\mathrm{vert}} = {\left[ 1, 0, 1 \right]}\) der Form \(1 \times 3\) angewandt. Dieser entfernt dünne vertikale Linien zwischen den einzelnen Fingern. Übrig bleibt das segmentierte Binärbild.

Für die Merkmalsextraktion werden aus dem segmentierten Binärbild Konturen extrahiert. Der Merkmalsvektor besteht aus den Komponenten:
%
\begin{description}
    \item[Fläche] Hauptargument für \defectok{} oder eine der anderen Defektklassen. Flächenmäßig kleine Konturen deuten eher auf \foreignlanguage{english}{false-Positives} der Segmentierung hin.
    \item[Ausrichtung] Unterscheidung von \defectfinger{} oder \defectcrack{}. Während stets vertikal (\(\theta = \frac{\pi}{2}\)) verlaufen, liegen \defectcrack{} oft schräg (\(\theta \neq \frac{\pi}{2}\)).
    \item[Umfang] Möglicherweise dienlich zur Unterteilung von \defectdarkarea{} und \defectfinger{}. Vermutlich aber das schwächste Merkmal in dieser Variante.
    \item[Exzentrizität] Unterscheidung von \defectcrack{} und \defectdarkarea{}, da erstere linienförmig und damit hoch exzentrisch auftreten.
\end{description}

Die Merkmalsextraktion endet mit der Auswahl des Merkmalsvektors mit der größten Flächenkomponente. Nur dieser wird für die Klassifikation des Gesamtbildes verwendet.

\begin{figure}[ht]
    \centering
    \begin{subfigure}{\textwidth}
        \centering
        \begin{tikzpicture}[
            start/.style={
                    font=\small,
                    draw,
                    very thick,
                    shape=ellipse,
                    text depth=0.35ex,
                    inner sep=0.25cm,
                    on chain,
                },
            end/.style={
                    start,
                    join,
                },
            process_unjoined/.style={
                    draw,
                    very thick,
                    rounded corners,
                    text width=2.5cm,
                    inner sep=0.25cm,
                    align=center,
                    on chain,
                },
            process/.style={
                    process_unjoined,
                    join,
                },
            connector/.style={very thick, -{Stealth[]}},
            every join/.style={connector},
            ]
            \begin{scope}[start chain=trunk going below, node distance=5mm]
                \node [start] {Start};
                \node [process] {Überbelichtung};
                \node (shading_process) [process] {Be\-lich\-tungs\-kor\-rek\-tur};
                \node (adapt_process) [process] {Adaptive Schwellwertbildung};
                \begin{scope}[start branch=canny going below]
                    \node (canny_process) [process_unjoined, right=of adapt_process] {Bus\-leis\-ten-Erkennung mittels Can\-ny Edge-Detector};
                    \node (bus_process) [process] {Maske: Busleisten};
                \end{scope}
                \begin{scope}[start branch=on going below]
                    \node (on_process) [process_unjoined, right=of canny_process] {Maske: Schwellwertbildung der \texttt{ON} Pixel};
                \end{scope}
                \begin{scope}[start branch=off going below]
                    \node (off_process) [process_unjoined, right=of on_process] {Maske: Schwellwertbildung der \texttt{OFF} Pixel};
                \end{scope}
                \path let \p{lowest}=(bus_process.south), \p{parent}=(adapt_process.south) in node (apply_process) [process] at (\x{parent}, \y{lowest}) {Anwendung der Bildmasken};
                \node [process] {Erodierung vertikaler Kanten};
                \node [end] {Ende};
            \end{scope}
            \draw [connector] (shading_process) -| (on_process);
            \draw [connector] (shading_process) -| (off_process);
            \draw [connector] (shading_process) -| (canny_process);
            \draw [connector] (on_process) |- (apply_process);
            \draw [connector] (off_process) |- (apply_process);
            \draw [connector] (bus_process) |- (apply_process);
        \end{tikzpicture}
        \caption{Segmentierung \textbf{A}.\label{fig:segmentation_a}}
    \end{subfigure}
    \\
    \begin{subfigure}{\textwidth}
        \centering
        \begin{tikzpicture}[
                feature/.style={
                        draw,
                        very thick,
                        text height=1.65ex,
                        text depth=0.35ex,
                        minimum width=2.5cm,
                        inner sep=0.25cm,
                    },
            ]
            \matrix [
                matrix of nodes,
                nodes={feature},
                node distance=1cm,
                row sep=0.25cm,
                column sep=0.25cm
            ] {
                Fläche & Ausrichtung   \\
                Umfang & Exzentrizität \\
            };
        \end{tikzpicture}
        \caption{Merkmalsextraktion \textbf{A}.\label{fig:feature_extraction_a}}
    \end{subfigure}
    \caption{Segmentierung und Merkmale von Variante 1.\label{fig:variant_1}}
\end{figure}

\begin{figure}[ht]
    \centering
    \includegraphics{images/bus_mask_finger.eps}
    \caption{Veranschaulichung der Busmasken Generierung.\label{fig:bus_mask_finger}}
\end{figure}

\begin{figure}[ht]
    \centering
    \includegraphics[width=\textwidth]{images/feature_space_var0.eps}
    \caption{Merkmalsraum der Variante 1\label{fig:feature_space_var0}}
\end{figure}

\begin{figure}[ht]
    \centering
    \includegraphics{images/confusion_matrix_var0.eps}
    \caption{\foreignlanguage{english}{Confusion Matrix} der Variante 1\label{fig:confusion_matrix_var0}}
\end{figure}

\subsection{Variante 2}

\subsection{Variante 3}

\subsection{Variante 4}
